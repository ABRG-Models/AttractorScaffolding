% This is an example of using latex for a paper/report of specified
% size/layout. It's useful if you want to provide a PDF that looks
% like it was made in a normal word processor.

% While writing, don't stop for errors
\nonstopmode

% Use the article doc class, with an 11 pt basic font size
\documentclass[11pt, a4paper]{article}

% Makes the main font Nimbus Roman, a Times New Roman lookalike:
%\usepackage{mathptmx}% http://ctan.org/pkg/mathptmx
% OR use this for proper Times New Roman (from msttcorefonts package
% on Ubuntu). Use xelatex instead of pdflatex to compile:
\usepackage{fontspec}
\usepackage{xltxtra}
\usepackage{xunicode}
\defaultfontfeatures{Scale=MatchLowercase,Mapping=tex-text}
\setmainfont{Times New Roman}

% Set margins
\usepackage[margin=2.5cm]{geometry}

% Multilingual support
\usepackage[english]{babel}

% Nice mathematics
\usepackage{amsmath}

% Control over maketitle
\usepackage{titling}

% Section styling
\usepackage{titlesec}

% Ability to use colour in text
\usepackage[usenames]{color}

% For the \degree symbol
\usepackage{gensymb}

% Allow includegraphics and nice wrapped figures
\usepackage{graphicx}
\usepackage{wrapfig}
\usepackage[outercaption]{sidecap}

% Set formats using titlesec
\titleformat*{\section}{\bfseries\rmfamily}
\titleformat*{\subsection}{\bfseries\itshape\rmfamily}

% thetitle is the number of the section. This sets the distance from
% the number to the section text.
\titlelabel{\thetitle.\hskip0.3em\relax}

% Set title spacing with titlesec, too.  The first {1.0ex plus .2ex
% minus .7ex} sets the spacing above the section title. The second
% {-1.0ex plus 0.2ex} sets the spacing the section title to the
% paragraph.
\titlespacing{\section}{0pc}{1.0ex plus .2ex minus .7ex}{-1.1ex plus 0.2ex}

%% Trick to define a language alias and permit language = {en} in the .bib file.
% From: http://tex.stackexchange.com/questions/199254/babel-define-language-synonym
\usepackage{letltxmacro}
\LetLtxMacro{\ORIGselectlanguage}{\selectlanguage}
\makeatletter
\DeclareRobustCommand{\selectlanguage}[1]{%
  \@ifundefined{alias@\string#1}
    {\ORIGselectlanguage{#1}}
    {\begingroup\edef\x{\endgroup
       \noexpand\ORIGselectlanguage{\@nameuse{alias@#1}}}\x}%
}
\newcommand{\definelanguagealias}[2]{%
  \@namedef{alias@#1}{#2}%
}
\makeatother
\definelanguagealias{en}{english}
\definelanguagealias{eng}{english}
%% End language alias trick

%% Any aliases here
\newcommand{\mb}[1]{\mathbf{#1}} % this won't work?
% Emphasis and bold.
\newcommand{\e}{\emph}
\newcommand{\mycite}[1]{\cite{#1}}
%% END aliases

% Custom font defs
% fontsize is \fontsize{fontsize}{linespacesize}
\def\authorListFont{\fontsize{11}{11} }
\def\corrAuthorFont{\fontsize{10}{10} }
\def\affiliationListFont{\fontsize{11}{11}\itshape }
\def\titleFont{\fontsize{14}{11} \bfseries }
\def\textFont{\fontsize{11}{11} }
\def\sectionHdrFont{\fontsize{11}{11}\bfseries}
\def\bibFont{\fontsize{10}{10} }
\def\captionFont{\fontsize{10}{10} }

% Caption font size to be small.
\usepackage[font=small,labelfont=bf]{caption}

% Make a dot for the dot product, call it vcdot for 'vector calculus
% dot'. Bigger than \cdot, smaller than \bullet.
\makeatletter
\newcommand*\vcdot{\mathpalette\vcdot@{.35}}
\newcommand*\vcdot@[2]{\mathbin{\vcenter{\hbox{\scalebox{#2}{$\m@th#1\bullet$}}}}}
\makeatother

\def\firstAuthorLast{James}

% Affiliations
\def\Address{\\
\affiliationListFont Adaptive Behaviour Research Group, Department of Psychology,
  The University of Sheffield, Sheffield, UK \\
}

% The Corresponding Author should be marked with an asterisk. Provide
% the exact contact address (this time including street name and city
% zip code) and email of the corresponding author
\def\corrAuthor{Seb James}
\def\corrAddress{Department of Psychology, The University of Sheffield,
  Western Bank, Sheffield, S10 2TP, UK}
\def\corrEmail{seb.james@sheffield.ac.uk}

% Figure out the font for the author list..
\def\Authors{\authorListFont Sebastian James\\[1 ex]  \Address \\
  \corrAuthorFont $^{*}$ Correspondence: \corrEmail}

% No page numbering please
\pagenumbering{gobble}

% A trick to get the bibliography to show up with 1. 2. etc in place
% of [1], [2] etc.:
\makeatletter
\renewcommand\@biblabel[1]{#1.}
\makeatother

% reduce separation between bibliography items if not using natbib:
\let\OLDthebibliography\thebibliography
\renewcommand\thebibliography[1]{
  \OLDthebibliography{#1}
  \setlength{\parskip}{0pt}
  \setlength{\itemsep}{0pt plus 0.3ex}
}

% Set correct font for bibliography (doesn't work yet)
%\renewcommand*{\bibfont}{\bibFont}

% No paragraph indenting to match the VPH format
\setlength{\parindent}{0pt}

% Skip a line after paragraphs
\setlength{\parskip}{0.5\baselineskip}
\onecolumn

% titling definitions
\pretitle{\begin{center}\titleFont}
\posttitle{\par\end{center}\vskip 0em}
\preauthor{ % Fonts are set within \Authors
        \vspace{-1.1cm} % Bring authors up towards title
        \begin{center}
        \begin{tabular}[t]{c}
}
\postauthor{\end{tabular}\par\end{center}}

% Define title, empty date and authors
\title {
  The combinatorics of fitness scores in an oscillating gene system
}
\date{} % No date please
\author{\Authors}

%% END OF PREAMBLE

\begin{document}

\setlength{\droptitle}{-1.8cm} % move the title up a suitable amount
\maketitle

\vspace{-1.8cm} % HACK bring the introduction up towards the title. It
                % would be better to do this with titling in \maketitle

\section{Introduction}

This is an attempt to describe the scoring system used to determine
the fitness of the gene system described in the paper ``How self
organization can guid natural selection''.

The paper describes a model system (or organism) with $n$ genes, each
of which contains the code from which a specific protein can be
created. Each of the $n$ genes has only two `individual gene
states'; \emph{expressed} (1) or \emph{dormant} (0). In the model, a
gene which is expressed is being used to create its corresponding
protein; a gene which is dormant is not.

The overall \emph{gene state} of the organism is given by all of the
possible individual gene state combinations of its $n$ genes. We can
write out the possible states in which a 3 gene system can exist; see
Table~\ref{tab:states}.

\begin{table}[h!]
  \begin{center}
    \caption{The $2^3 = 8$ states of a 3 gene system.}
    \label{tab:states}
    \begin{tabular}{c|c}
      \textbf{State (decimal)} & \textbf{State (binary)} \\
      \hline
      0 & 0 0 0\\
      1 & 0 0 1\\
      2 & 0 1 0\\
      3 & 0 1 1\\
      4 & 1 0 0\\
      5 & 1 0 1\\
      6 & 1 1 0\\
      7 & 1 1 1\\
    \end{tabular}
  \end{center}
\end{table}

In the gene state 0~0~0, none of the three genes are expressed; in the
state 1~1~1, all three genes are coding for their respective
proteins. There are 8 gene states ($2^3$) for this 3 gene system and
$2^n$ states for an $n$ gene system.

In addition to this gene state space, the model describes
a \emph{genome state space}.

[Describe genome state space, so that reader knows that gene states
transition and arrive in attractor limit cycles, which are compared
with target states.]

\section{Fitness scores}

The fitness of a limit cycle of size $l$, containing the states $s_j$
where $j=1,2,...,l$, is determined by comparing each $s_j$ in the
limit cycle to a desired target state, $t$. The fitness, $F$ is given
by:
%
\begin{equation}\label{eq:fitness}
F = \prod_{i=1}^{n} \frac{ \sum_{j=1}^{l} !(s_i \oplus t_i)}{l}
\end{equation}

To understand this, it's best to write out an example in a
table. Table~\ref{tab:scoring} is a 3 gene example of a limit cycle
being compared to a target state $t$ = 1~0~1.

\begin{table}[h!]
  \begin{center}
    \caption{Deconstructing Eq.~\ref{eq:fitness}; scoring the fitness of a 3 gene ($n=3$) limit cycle of
    length 3 ($l=3$). The fitness, $F$=2/27.}
    \label{tab:scoring}
    \begin{tabular}{c|c|c|c}
      \textbf{Target ($t$)} & \textbf{Limit cycle ($s_i$)} & Compare
    $t$ and $s_i$: \textbf{$!(s_{i,j} \oplus t_j)$} \\
      \hline
      1 0 1 & 0 0 1 & 0 1 1 &\\
            & 0 1 0 & 0 0 0 &\\
            & 1 1 1 & 1 0 1 &\\
      \hline
      & & $\sum_{j=1}^{l} !(s_{i,j} \oplus t_j)$ & \\
      \hline
      & & 1 1 2 & \\
      \hline
      & & ${\sum_{j=1}^{l} !(s_{i,j} \oplus t_j)} \div {(l=3)}$ & \prod_{i=1}^{n}\\
      \hline
      & & $\frac{1}{3}$
    $\frac{1}{3}$ $\frac{2}{3}$ &
    $\frac{1}{3} \times \frac{1}{3} \times \frac{2}{3} = \frac{2}{27}$ \\
    \end{tabular}
  \end{center}
\end{table}

\section{Probability of zero score}

The probability of having a zero score in a limit cycle is the same as
the probability of having ALL zeros in any column of $!(s_{i,j} \oplus
t_j)$.

I'll use the following notation:

$P(ZC_0)$ is the probability of having an all-zero column 0 (counting
columns up from 0). $P(!ZC_0) is the probability of \emph{not} having
all zeros in column 0. $P(ZC_1|!ZC_0) is the probability of having all
zeros in column 1 given that there are not all zeros in column 0.

For an $n=3$ gene system, the probability of a zero score, $P(0)$, is:
\begin{equation} \label{eq:pzero}
P(0) = P(ZC_0) + P(ZC_1\;|\;!ZC_0) \times P(ZC_1) + P(ZC_2\;|\;!ZC_0\;\land\;!ZC_1) \times P(ZC_2)
\end{equation}

It's possible to determine these probabilities for systems of any
number of genes, $n$ and any size limit cycle, $l\leq n $.

\subsection{The probability of there being a zero in the first column}

The first column is easy. We simply consider the number of possible
sets of states that can make up a limit cycle of size $l$, and
determine which of those have all zeros in the first column. Note that
it's not actually the \emph{states} in the limit cycle which give the
score, but the results of XNORing the states with the target. However,
there are $2^n$ unique states $s_i$ for $n$ genes, and also exactly
$2^n$ unique results of $!(t \oplus s_i)$, so the probabilities can be
determined by considering the number of possible sets of states in the
limit cycle.

Here, I've introduced the term \emph{a set of states} or \emph{state
set}, referring to the unique set of states which a limit cycle
contains. Note that a limit cycle is a mathematical set; no state in
the set may appear more than once. For this reason, there is a limited
number of possible state combinations for systems of $n$ genes that
can make up a limit cycle of a given length $l$, and it is given by
the binomial theorem:
\begin{equation}
\mathrm{The~number~of~possible~state~sets} = \binom{2^n}{l}
\end{equation}

Thinking of $n=3$ again, look at Table~\ref{tab:states}. Four of the
eight possible states have a zero in the zeroth column. So, the number
of ways to arrange $l$ states such that there are all zeros in the
zeroth column is $\binom{4}{l}$. If $l=2$, that's
$\binom{4}{2}=6$. All the $l=2$ limit cycle state sets for $n=3$ genes
are shown in Table~\ref{tab:n3l2}. Inspection of this table shows that
there are 6 state sets out of 28 which have all zeros in the zeroeth
column. In general,

\begin{equation}
\mathrm{The~number~of~state~sets~with~all~zeros~in~one~column} = \binom{2^n/2}{l}
\end{equation}


\begin{table}[h!]
  \begin{center}
    \caption{All of the possible $l=2$ sized limit cycle state sets formed from
    $n=3$ genes. Where there are all zeros in the zeroeth column, they
    have been typeset in bold.}
    \label{tab:n3l2}
    \begin{tabular}{c|c|c|c|c|c}
      \textbf{ss0} & \textbf{ss1} & \textbf{ss2} & \textbf{ss3} &\textbf{ss4} & \textbf{ss5} \\
      \hline
      \textbf{0} 0 0       & \textbf{0} 0 0       & \textbf{0} 0 0       & 0 0 0       & 0 0 0      & 0 0 0       \\
      \textbf{0} 0 1       & \textbf{0} 1 0       & \textbf{0} 1 1       & 1 0 0       & 1 0 1      & 1 1 0       \\
      \hline
      \textbf{ss6} & \textbf{ss7} & \textbf{ss8} & \textbf{ss9} &\textbf{ss10} & \textbf{ss11} \\
      \hline
      0 0 0       & \textbf{0} 0 1       & \textbf{0} 0 1       & 0 0 1       & 0 0 1      & 0 0 1       \\
      1 1 1       & \textbf{0} 1 0       & \textbf{0} 1 1       & 1 0 0       & 1 0 1      & 1 1 0       \\
      \hline
      \textbf{ss12} & \textbf{ss13} & \textbf{ss14} & \textbf{ss15} &\textbf{ss16} & \textbf{ss17} \\
      \hline
      0 0 1       & \textbf{0} 1 0       & 0 1 0       & 0 1 0       & 0 1 0      & 0 1 0       \\
      1 1 1       & \textbf{0} 1 1       & 1 0 0       & 1 0 1       & 1 1 0      & 1 1 1       \\
      \hline
      \textbf{ss18} & \textbf{ss19} & \textbf{ss20} & \textbf{ss21} &\textbf{ss22} & \textbf{ss23} \\
      \hline
      0 1 1       & 0 1 1       & 0 1 1       & 0 1 1       & 1 0 0      & 1 0 0       \\
      1 0 0       & 1 0 1       & 1 1 0       & 1 1 1       & 1 0 1      & 1 1 0       \\
      \hline
      \textbf{ss24} & \textbf{ss25} & \textbf{ss26} & \textbf{ss27} &        &             \\
      \hline
      1 0 0       & 1 0 1       & 1 0 1       & 1 1 0       &            &             \\
      1 1 1       & 1 1 0       & 1 1 1       & 1 1 1       &            &             \\
      \hline
    \end{tabular}
  \end{center}
\end{table}

So, finally, we can write down the probability of having all zeros in
the zeroeth column for $n$ genes forming a limit cycle of length $l$ as:
\begin{equation}
P(ZC_0) = \frac{\binom{2^n/2}{l}}{\binom{2^n}{l}}
\end{equation}

\subsection{The probability of there being all zeros in subsequent columns}

The probability of there being all zeros in the \emph{first}
column, \emph{given that} there are \emph{not} all zeros in the
zeroeth column is a problem of dependent probability. Refer to
Table~\ref{tab:n3l2}; ss0, ss3, ss4, ss9, ss10 and ss22 have all zeros
in the second column. The probability of finding all zeros in the
first column depends on whether or not all zeros have been found in
the zeroeth column. If we are given that the zeroeth column does not
contain all zeros, then there are only 5 possible state sets that can
fulfil this requirement; ss0 has zeros in the first column \emph{and
also in the zeroth}. The states that we must be drawing from cannot
include those for which there are all zeros in the zeroeth column. The
probability is therefore:

\begin{equation}
P(ZC_1|!ZC_0) = \frac{6 -1}{28-6} = \frac{5}{22}
\end{equation}

A general expression for the probability of there being all zeros in a
column $m$, given that there are NOT all zeros in columns preceding
$m$ is:

\begin{equation}
  P(ZC_m|!ZC_{0\rightarrow[m-1]}) = \begin{cases}
    \frac
        {\binom{2^n/2}{l} }
        {\binom{2^n}{l}}                                                             & m=0 \\
        & \\
    \frac
        {\binom{2^n/2}{l} - \left( \sum_{i=1}^m \binom{m}{i} \binom{2^{n-1}/2^i}{l}(-1)^{i+1} \right)}
        {\binom{2^n}{l} - \left( \sum_{i=1}^m \binom{m}{i} \binom{2^n/2^i}{l}(-1)^{i+1} \right)}    &  m>0 \\
    \end{cases}
\end{equation}

Compute this for $m=0$ up to $m=n-1$ to be able to compute $P(0)$ and $P(!0)$ using a version of Eq.~\ref{eq:pzero}.

%
% BIBLIOGRAPHY
%
\selectlanguage{English}
\bibliographystyle{abbrvnotitle}
% The bibliography NoTremor.bib is the one exported from Zotero. It
% may be necessary to run my UTF-8 cleanup script, bbl_utf8_to_latex.sh
%%\bibliography{NoTremor}

%%% Upload the *bib file along with the *tex file and PDF on
%%% submission if the bibliography is not in the main *tex file

\end{document}
